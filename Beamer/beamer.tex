\documentclass{beamer}
\usepackage[utf8]{inputenc}
\usetheme{Warsaw}
\usecolortheme{crane}
\usecolortheme{orchid}

\title{Package Coberny}
\author{A.Bernard, F.Chery, O.Côme}\institute{Faculté des sciences de Montpellier}
\date{13 Décembre 2021}

\begin{document}

\begin{frame}
\titlepage
\end{frame}

\begin{frame}
  \frametitle{Sommaire}
  \tableofcontents
\end{frame}

\section{Introduction}
\subsection{Création de la base de données}

\begin{frame}[fragile]{Création de la base de données}
Dataframe intermédiaire : \newline
\begin{itemize}
\item Pour créer le data nous avons utilisé $\textbf{pandas}$ pour sélectionner uniquement les sorties d'autoroute concernées par le projet et enlever les portions gratuites.  
\item Nous avons utilisé $\textbf{pyproj}$ pour transformer les coordonnées L93 en WGS84. Nous avons donc obtenu à la suite un dataframe avec les noms des autoroutes, les noms des péages et les coordonnées GPS.
\end{itemize}
\end{frame}

\begin{frame}[fragile]{Création de la base de données}
Dataframe des distances et des prix: \newline
\begin{itemize}
\item Nous avons utilisé $\textbf{requests}$ et $\textbf{json}$ pour faire des requêtes de distance entre chaque coordonnées du dataframe créé précédemment. Ces packages utilisent les données de $\textbf{openstreetmap}$.
\item Nous avons simplement reporté le fichier que nous avions en fichier $\textit{.csv}$ pour l'utiliser avec $\textbf{pandas}$ et choisir les péages voulus. Puis nous avons renommé les colonnes pour être cohérent avec les autres dataframe.
\end{itemize}
\end{frame}


\subsection{Présentation du package Coberny}

\begin{frame}[fragile]{Présentation du package Coberny}
\begin{block}{Installer Coberny}
pip install git+https://github.com/ABernard27/PROJET-groupe-3
\end{block}
Ce package permet de réaliser 3 actions primaires : \newline
\begin{itemize}
\item Réaliser une carte intéractive d'un trajet sur l'autoroute en affichant les noms des stations, le prix entre deux stations et le temps en kilomètres.
\item Afficher la distribution des prix entre deux sorties
\item Déterminer, en fonction du nombre de sorties acceptées, le trajet le moins coûteux
\end{itemize}
\end{frame}



\section{Carte intéractive}
\subsection{Création de la carte}

\begin{frame}[fragile]{Création de la carte}
\end{frame}

\subsection{Utilisation}
\begin{frame}[fragile]{Utilisation}
\end{frame}


\section{Distribution des prix}
\begin{frame}
nzkldnz
\end{frame}

\section{Minimisation coût du trajet}
\begin{frame}
zdn kzlf dz
\end{frame}



\end{document}



