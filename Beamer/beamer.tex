\documentclass{beamer}
\usepackage[utf8]{inputenc}
\usetheme{Warsaw}
\usecolortheme{crane}
\usecolortheme{orchid}

\title{Package Coberny}
\author{A.Bernard, F.Chery, O.Côme}\institute{Faculté des sciences de Montpellier}
\date{13 Décembre 2021}

\begin{document}

\begin{frame}
\titlepage
\end{frame}

\begin{frame}
  \frametitle{Sommaire}
  \tableofcontents
\end{frame}

\section{Introduction}
\subsection{Création de la base de données}

\begin{frame}[fragile]{Création de la base de données}
Pour le dataframe des distances : \newline
\begin{itemize}
\item Création avec $\textbf{pandas}$ des dataframe en selectionnant uniquement les sorties d'autoroute concernées par le projet et en enlevant les portions gratuites
\item Transformation s
\end{itemize}
\end{frame}


\subsection{Présentation du package Coberny}

\begin{frame}[fragile]{Présentation du package Coberny}
\begin{block}{Installer Coberny}
pip install "à compléter"
\end{block}
Ce package permet de réaliser 3 actions primaires : \newline
\begin{itemize}
\item Réaliser une carte intéractive d'un trajet sur l'autoroute en affichant les noms des stations, le prix entre deux stations et le temps en kilomètres.
\item Afficher la distribution des prix entre deux sorties
\item Déterminer, en fonction du nombre de sorties acceptées, le trajet le moins coûteux
\end{itemize}
\end{frame}


\section{Carte intéractive}
\begin{frame}
\begin{block}{Installation}
blablablabla
\end{block}
\end{frame}


\section{Distribution des prix}
\begin{frame}
nzkldnz
\end{frame}

\section{Minimisation coût du trajet}
\begin{frame}
zdn kzlf dz
\end{frame}



\end{document}



